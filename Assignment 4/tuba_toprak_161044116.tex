\documentclass[a4 paper]{article}
\usepackage[inner=2.0cm,outer=2.0cm,top=2.5cm,bottom=2.5cm]{geometry}
\usepackage{setspace}
\usepackage[rgb]{xcolor}
\usepackage{verbatim}
\usepackage{subcaption}
\usepackage{amsgen,amsmath,amstext,amsbsy,amsopn,tikz,amssymb}
\usepackage[colorlinks=true, urlcolor=blue,  linkcolor=blue, citecolor=blue]{hyperref}
\usepackage[colorinlistoftodos]{todonotes}
\usepackage{rotating}
\usepackage{booktabs}
\newcommand{\ra}[1]{\renewcommand{\arraystretch}{#1}}

\newtheorem{thm}{Theorem}[section]
\newtheorem{prop}[thm]{Proposition}
\newtheorem{lem}[thm]{Lemma}
\newtheorem{cor}[thm]{Corollary}
\newtheorem{defn}[thm]{Definition}
\newtheorem{rem}[thm]{Remark}
\numberwithin{equation}{section}

\newcommand{\homework}[6]{
   \pagestyle{myheadings}
   \thispagestyle{plain}
   \newpage
   \setcounter{page}{1}
   \noindent
   \begin{center}
   \framebox{
      \vbox{\vspace{2mm}
    \hbox to 6.28in { {\bf CSE 211:~Discrete Mathematics \hfill {\small (#2)}} }
       \vspace{6mm}
       \hbox to 6.28in { {\Large \hfill #1  \hfill} }
       \vspace{6mm}
       \hbox to 6.28in { {\it Instructor: {\rm #3} \hfill Name:TUBA TOPRAK {\rm #5} \hfill Student Id: 161044116{\rm #6}} \hfill}
       \hbox to 6.28in { {\it Assistant: #4  \hfill #6}}
      \vspace{2mm}}
   }
   \end{center}
   \markboth{#5 -- #1}{#5 -- #1}
   \vspace*{4mm}
}

\newcommand{\problem}[2]{~\\\fbox{\textbf{Problem #1}}\hfill (#2 points)\newline\newline}
\newcommand{\subproblem}[1]{~\newline\textbf{(#1)}}
\newcommand{\D}{\mathcal{D}}
\newcommand{\Hy}{\mathcal{H}}
\newcommand{\VS}{\textrm{VS}}
\newcommand{\solution}{~\newline\textbf{\textit{(Solution)}} }

\newcommand{\bbF}{\mathbb{F}}
\newcommand{\bbX}{\mathbb{X}}
\newcommand{\bI}{\mathbf{I}}
\newcommand{\bX}{\mathbf{X}}
\newcommand{\bY}{\mathbf{Y}}
\newcommand{\bepsilon}{\boldsymbol{\epsilon}}
\newcommand{\balpha}{\boldsymbol{\alpha}}
\newcommand{\bbeta}{\boldsymbol{\beta}}
\newcommand{\0}{\mathbf{0}}


\begin{document}
\homework{Homework \#4}{Due: 17/01/21}{Dr. Zafeirakis Zafeirakopoulos}{Gizem S\"ung\"u}{}{}
\textbf{Course Policy}: Read all the instructions below carefully before you start working on the assignment, and before you make a submission.
\begin{itemize}
	\item It is not a group homework. Do not share your answers to anyone in any circumstance. Any cheating means at least -100 for both sides. 
	\item Do not take any information from Internet.
	\item No late homework will be accepted. 
	\item For any questions about the homework, send an email to gizemsungu@gtu.edu.tr
	\item The homeworks (both latex and pdf files in a zip file) will be
	submitted into the course page of Moodle.
	\item The latex, pdf and zip files of the homeworks should be saved as
	"Name\_Surname\_StudentId".$\{$tex, pdf, zip$\}$.
	\item If the answers of the homeworks have only calculations without any formula or any explanation -when needed- will get zero.
	\item Writing the homeworks on Latex is strongly suggested. However, hand-written paper is still accepted $\textbf{IFF}$ hand writing of the student is clear and understandable to read, and the paper is well-organized. Otherwise, the assistant cannot grade the student's homework.
\end{itemize}

\problem{1}{15+15=30}
Consider the nonhomogeneous linear recurrence relation $a_n$ = 3$a_{n-1}$ + $2^n$ .\\
\subproblem{a} Show that whether $a_n$ = $-2^{n+1}$ is a solution of the given recurrence relation or not. Show your work step by step.
\solution
\newline
\newline
$a_n$ = $-2^{n+1}$
\newline
$a_{n-1}$ = $-2^{(n-1)+1}$
\newline
$a_{n-1}$ = $-2^{n}$
\newline
Now, we can write the expression of $a_{n-1}$ to check that $a_n$ = $-2^{n+1}$ is a solution of our recurrence relation or not.
\newline
$a_n$ = $3a_{n-1} + 2^n$
\newline
$a_n$ = $3(-2^n) + 2^n$
\newline
$a_n$ = $-2.2^n$
\newline
$a_n$ = $-2^{n+1}$
\newline
Then we can say that, $a_n$ = $-2^{n+1}$ is a solution of the $a_n$ = $3a_{n-1} + 2^n$
\newline
\subproblem{b} Find the solution with $a_0$ = 1.
\solution
\newline
It's a non-homogeneous recurrence relation, so we can define it like,
\newline
$a_n^{(g)}$ = $a_n^{(h)} + a_n^{(p)}$
\newline
Firstly we should find the homogeneous part to write the characteristic equation.
\newline
$a_{n-1} = 1$
\newline
$a_n = r$
\newline
$a_n$ = $3a_{n-1}$(homogeneous part of the given recurrence relation)
\newline
r = 3 (characteristic root)
\newline
$a_n^{(h)}$ = $c_1.3^n$
\newline
to find the particular part of the given recurrence relation,
\newline
$a_n^{(p)}$ = $A.2^n$
\newline
$a_{n-1}^{(p)}$ = $A.2^{n-1}$
\newline
$a_n$ = $3a_{n-1} + 2^n$ (The given recurrence relation)
\newline
$A.2^n$ = $3.A.2^{n-1} + 2^n$
\newline
= $((3.A)/2).2^n + 2^n$
\newline
$A.2^n$ = $2^n.(((3.A)/2)+1)$
\newline
$(((3.A)/2)+1)$ = A
\newline
2.A - 2 = 3.A
\newline
A = -2
\newline
$a_n^{(p)}$ = $A.2^n$ = $-2.2^n$ = $-2^{n+1}$
\newline
$a_n^{(g)}$ = $a_n^{(h)} + a_n^{(p)}$
\newline
$a_n^{(g)}$ = $c_1.3^n-2^{n+1}$
\newline
if $a_0 = 1$ then $a_0$ = $c_1-2$ = 1
\newline
$c_1$ = 3 so,
\newline
$a_n$ = $3^{n+1}-2^{n+1}$
\newline
\problem{2}{35}
Solve the recurrence relation f(n) = 4f(n-1) - 4f(n-2) + $n^2$ for f(0) = 2 and f(1) = 5. 
\solution
\newline
linear nonhomogeneous recurrence with constant coefficient
\newline
$f_n^{(g)}$ = $f_n^{(h)} + f_n^{(p)}$
\newline
$f_n^{(h)}$Associated homogeneous recurrence relation is an = 4f(n−1)- 4f(n−2). 
\newline
Characteristic equation: r 2 − 4r + 4 = 0  Factor. r 2 − 4r + 4 = 0
\newline 
(r − 2)(r − 2)= 0 r = 2
\newline
$f_n^{(h)} =c1(2)^{n}+ c2(2)^{n}.n$
\newline
$f_n^{(h)} =2^{n}(c1+c2.n)$
\newline
$f_n^{(p)} =n^{2}+8n+20$
\newline
$f_n^{(g)}$ = $f_n^{(h)} + f_n^{(p)}$
\newline
$f_n =2^{n}( c1+ c2.n) +n^{2}+8n+20 $
\newline
$f(0) =2^{0}( c1+ c2.0) +0^{2}+8.0+20 = 2 $
\newline
c1+20 = 2
\newline
c1 = -18
 \newline
$f(1) =2^{1}( c1+ c2.1) +1^{2}+8.1+20  = 5$
 \newline
 $f(1) =2(c1 +c2)+29  = 5 $
 \newline
  $f(1) =2(-18 +c2)+29  = 5 $
  \newline
  c2 = 6
\newline
$f_n =2^{n}( 6.n - 18) +n^{2}+8n+20 $
\newpage
\problem{3}{20+15 = 35}
Consider the linear homogeneous recurrence relation $a_n$ = 2$a_{n-1}$ - 2$a_{n-2}$.
\subproblem{a} Find the characteristic roots of the recurrence relation.
\solution
\newline
It's homogeneous , so we can define it like this,
\newline
$a_n$ = $a_n^{(h)}$
\newline
To find the characteristic equation.
\newline
$a_{n-2}$ = 1
\newline
$a_{n-1}$ = r
\newline
$a_n$ = $r^2$
\newline
$a_n$ = 2$a_{n-1}$ - 2$a_{n-2}$ (The given recurrence relation)
\newline
$r^2$ = 2r-2
\newline
$r^2$-2r+2 = 0
\newline
We will use discriminant to find the characteristic roots.
\newline
$\Delta$ = $b^2$ - 4.a.c
\newline
$\Delta$ = $-2^2$ - 4.(1).(2) = 4 - 8 = -4
\newline
$\Delta$ $<$ 0 (It means that there is no root in real numbers)
\newline
$r_1$ = -b+$\sqrt\Delta$/(2.a) = -(-2)+$\sqrt-4$/(2.1) = 2+$\sqrt-4$/2 = (2 + 2i)/2 = 1 + i
\newline
$r_2$ = -b-$\sqrt\Delta$/(2.a) = -(-2)-$\sqrt-4$/(2.1) = 2-$\sqrt-4$/2 = (2 - 2i)/2 = 1 - i
\newline
The characteristic roots are 1 + i and 1 - i
\newline
\newline
\subproblem{b} Find the solution of the recurrence relation with $a_0$ = 1 and $a_1$ = 2.
\solution
\newline
We should write its general equation like this,
\newline
$a_n^{(h)}$ = $c_1$.$r_1^n$ + $c_2$.$r_2^n$ 
\newline
$a_n$ = $c_1$.$(1 + i)^n$ + $c_2$.$(1 - i)^n$ 
\newline
Now, we can use the $a_0$ = 1 and $a_1$ = 2 to find the values of $c_1$ and  $c_2$.
\newline
Let's start with $a_0$ = 1
If n = 0, then $c_1$ + $c_2$ = 1.
\newline
for $a_1$ = 2
\newline
$a_n$ = $c_1.(1 + i)^n$ + $c_2.(1 - i)^n$
\newline
$a_1$ = $c_1 + i.c_1 + c_2 - i.c_2$ = 2
\newline
we already know that,
\newline
$c_1 + c_2$ = 1
\newline
1 + i($c_1 - c_2$) = 2
\newline
$c_1 - c_2$ = 1/i = $i^{-1}$
\newline
$c_1 + c_2$ = 1
\newline
$c_1 - c_2$ = $i^{-1}$
\newline
$2c_1$ = $1 + i^{-1}$ so $c_1$ = $(1 + i^{-1})/2$
\newline
$c_2$ = $1 - ((1 + i^{-1})/2)$ so $c_2$ = $(1 - i^{-1})/2$
\newline
As a conclusion, the solution of the recurrence relation with $a_0$ = 1 and $a_1$ = 2 is,
\newline
$a_n$=$((1 + i^{-1})/2)$.$(1 + i)^n$ + $((1 - i^{-1})/2)$.$(1 - i)^n$ 
\newline
\end{document} 


