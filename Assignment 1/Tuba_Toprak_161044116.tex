\documentclass[a4 paper]{article}
\usepackage[inner=2.0cm,outer=2.0cm,top=2.5cm,bottom=2.5cm]{geometry}
\usepackage{setspace}
\usepackage[rgb]{xcolor}
\usepackage{verbatim}
\usepackage{subcaption}
\usepackage{amsgen,amsmath,amstext,amsbsy,amsopn,tikz,amssymb}
\usepackage{fancyhdr}
\usepackage[colorlinks=true, urlcolor=blue,  linkcolor=blue, citecolor=blue]{hyperref}
\usepackage[colorinlistoftodos]{todonotes}
\usepackage{rotating}
\usepackage{booktabs}
\newcommand{\ra}[1]{\renewcommand{\arraystretch}{#1}}

\newtheorem{thm}{Theorem}[section]
\newtheorem{prop}[thm]{Proposition}
\newtheorem{lem}[thm]{Lemma}
\newtheorem{cor}[thm]{Corollary}
\newtheorem{defn}[thm]{Definition}
\newtheorem{rem}[thm]{Remark}
\numberwithin{equation}{section}

\newcommand{\homework}[6]{
   \pagestyle{myheadings}
   \thispagestyle{plain}
   \newpage
   \setcounter{page}{1}
   \noindent
   \begin{center}
   \framebox{
      \vbox{\vspace{2mm}
    \hbox to 6.28in { {\bf CSE 211:~Discrete Mathematics \hfill {\small (#2)}} }
       \vspace{6mm}
       \hbox to 6.28in { {\Large \hfill #1  \hfill} }
       \vspace{6mm}
       \hbox to 6.28in { {\it Instructor: {\rm #3} \hfill  {\rm #5} \hfill  {\rm #6}} \hfill}
       \hbox to 6.28in { {\it Assistant: #4  \hfill #6}}
      \vspace{2mm}}
   }
   \end{center}
   \markboth{#5 -- #1}{#5 -- #1}
   \vspace*{4mm}
}

\newcommand{\problem}[2]{~\\\fbox{\textbf{Problem #1}}\hfill (#2 points)\newline\newline}
\newcommand{\subproblem}[1]{~\newline\textbf{(#1)}}
\newcommand{\D}{\mathcal{D}}
\newcommand{\Hy}{\mathcal{H}}
\newcommand{\VS}{\textrm{VS}}
\newcommand{\solution}{~\newline\textbf{\textit{(Solution)}} }

\newcommand{\bbF}{\mathbb{F}}
\newcommand{\bbX}{\mathbb{X}}
\newcommand{\bI}{\mathbf{I}}
\newcommand{\bX}{\mathbf{X}}
\newcommand{\bY}{\mathbf{Y}}
\newcommand{\bepsilon}{\boldsymbol{\epsilon}}
\newcommand{\balpha}{\boldsymbol{\alpha}}
\newcommand{\bbeta}{\boldsymbol{\beta}}
\newcommand{\0}{\mathbf{0}}


\begin{document}
\homework{Homework \#1}{Due: 17/11/20}{Dr. Zafeirakis Zafeirakopoulos}{Gizem S\"ung\"u}{}{}
\textbf{Course Policy}: Read all the instructions below carefully before you start working on the assignment, and before you make a submission.
\begin{itemize}
\item It is not a group homework. Do not share your answers to anyone in any circumstance. Any cheating means at least -100 for both sides. 
\item Do not take any information from Internet.
\item No late homework will be accepted. 
\item For any questions about the homework, send an email to gizemsungu@gtu.edu.tr
\item The homeworks (both latex and pdf files in a zip file) will be
submitted into the course page of Moodle.
\item The latex, pdf and zip files of the homeworks should be saved as
"Name\_Surname\_StudentId".$\{$tex, pdf, zip$\}$.
\item If the answers of the homeworks have only calculations without any formula or any explanation -when needed- will get zero.
\item Writing the homeworks on Latex is strongly suggested. However, hand-written paper is still accepted $\textbf{IFF}$ hand writing of the student is clear and understandable to read, and the paper is well-organized. Otherwise, the assistant cannot grade the student's homework.
\end{itemize}

\problem{1: Conditional Statements}{5+5+5=15}
State the converse, contrapositive, and inverse of each of these conditional statements.


\subproblem{a} If it snows tonight, then I will stay at home.
\solution
\newline
\textbf{Converse:}
If I will stay at home,then it snows tonight.
\newline
\textbf{Contrapositive:} If I will not stay at home,then it does not snow tonight.
\newline
\textbf{Inverse:} If it does not snows tonight, then I will not stay at home.
\newline
\subproblem{b} I go to the beach whenever it is a sunny summer day.
\solution
\newline
\textbf{Converse:} it is a sunny summer day whenever I go to the beach.
\newline
\textbf{Contrapositive:}it is not a sunny summer day whenever I do not go to the beach.
\newline
\textbf{Inverse:}I don't go to the beach whenever it is not a sunny summer day.
\newline

\subproblem{c} If I stay up late, then I sleep until
noon.
\solution
\newline
\newline
\newline
\textbf{Converse:} When I sleep until noon,it is necessary that I stay up late.
\newline
\newline
\textbf{Contrapositive:}When I don't sleep until noon,it is necessary that I don't stay up late
\newline
\newline
\textbf{Inverse:}When I do not stay up late,it is necessary that I do not sleep until noon
\newline
\newline


\problem{2: Truth Tables For Logic Operators}{5+5+5=15}
Construct a truth table for each of the following compound propositions.
\subproblem{a} (p $\oplus$ $\neg$ q)
\solution
\begin{center}\begin{tabular}{|l|l|l|}
\hline
p & $\neg$ q & p $\oplus$ $\neg$ q \\
\hline
1 & 1 & 0 \\
\hline
1 & 0 & 1 \\
\hline
0 & 1 & 1 \\
\hline
0 & 0 & 0 \\
\hline
\end{tabular}\end{center}

\subproblem{b} (p $\iff$ q) $\oplus$ ( $\neg$ p $\iff$ $\neg$ r)
\solution 
\begin{center}\begin{tabular}{|l|l|l|l|l|l|}
\hline
p & q & r & $\neg$ r & $\neg$ p & (p $\iff$ q) $\oplus$ ( $\neg$ p $\iff$ $\neg$ r) \\
\hline
 1 & 1 & 1 & 0 & 0 & 0 \\
\hline
 1 & 1 & 0 & 1 & 0 & 1 \\
\hline
 1 & 0 & 1 & 0 & 0 & 1 \\
\hline
 1 & 0 & 0 & 1 & 0 & 0 \\
\hline
 0 & 1 & 1 & 0 & 1 & 0 \\
\hline
 0 & 1 & 0 & 1 & 1 & 1 \\
\hline
 0 & 0 & 1 & 0 & 1 & 1 \\
\hline
 0 & 0 & 0 & 1 & 1 & 1 \\
\hline
\end{tabular}\end{center}

\subproblem{c} (p $\oplus$ q) $\Rightarrow$ (p $\oplus$ $\neg$ q)
\solution
\begin{center}\begin{tabular}{|l|l|l|l|}
\hline
p & q &$\neg$ q &(p $\oplus$ q) $\Rightarrow$ (p $\oplus$ $\neg$ q) \\
\hline
1 & 1 & 0 & 1 \\
\hline
1 & 0 & 1 & 0 \\
\hline
0 & 1 & 0 & 0 \\
\hline
0 & 0 & 1 & 1 \\
\hline
\end{tabular}\end{center}
\newpage


\problem{3: Predicates and Quantifiers}{21}
There are three predicate logic statements which represent English sentences as follows.

\begin{itemize}
	\item P(x): "x can speak English."
	\item Q(x): "x knows Python."
	\item H(x): "x is happy."
\end{itemize}

Express each of the following sentences in terms of P(x), Q(x), H(x), quantifiers, and logical connectives or vice versa. The domain
for quantifiers consists of all students at the university.

\subproblem{a} There is a student at the university who can speak English and who knows Python.
\solution
\newline
$\exists x(P(x) \wedge Q(x)) $
\newline
\subproblem{b} There is a student at the university who can speak English but who doesn’t know Python.
\solution
\newline
$\exists x(P(x)\wedge \neg Q(x))$
\newline
\subproblem{c} Every student at the university either can speak English or knows Python.
\solution
\newline
$\forall x(P(x) \vee Q(x)) $
\newline
\subproblem{d} No student at the university can speak English or knows Python.
\solution
\newline
$\forall x\neg(P(x) \vee Q(x)) $
\newline
\subproblem{e} If there is a student at the university who can speak English and know Python, then she/he is happy.
\solution
\newline
$\exists xP(x) \land \exists xQ(x) \to \exists xH(x)  $
\subproblem{f} At least two students are happy.
\solution
\newline
$\exists xH(x)$
\subproblem{g} $\neg \forall x (Q(x) \wedge P(x))$
\solution
\newline
No every student at the university  who can speak English and who knows Python
\newline

\problem{4: Mathematical Induction}{21}
Prove that 3 + 3 . 5 + 3 . $5^2$ + . . . + 3 . $5^n$ =$\frac{3(5^{n+1} - 1)}{4}$
whenever n is a nonnegative integer.
\solution
\newline
basic step n = 0
\newline
3.$5^0$  = 3
\newline
$\frac{3(5^{0+1} - 1)}{4}$ =$\frac{3(5^{1} - 1)}{4}$ = 3
\newline
see  P(0) is true for both sides of the equations.
\newline
Induction step = let P(k) be true.
\newline
3 + 3 . 5 + 3 . $5^2$ + . . . + 3 . $5^k$ =$\frac{3(5^{k+1} - 1)}{4}$
\newline
let P(k+1) be true.
\newline
3 + 3 . 5 + 3 . $5^2$ + . . . + 3. $5^k$ + 3 . $5^{k+1}$ =$\frac{3(5^{k+1} - 1)}{4}$ + 3. $5^{k+1}$ 
\newline
=3 . $ (\frac{(5^{k+1} - 1)}{4}$ +  $5^{k+1}$ )
\newline
=$ \frac{3}{4}$ . $ (\frac{(5^{k+1} - 1)}{4}$ + 4. $5^{k+1}$ )
\newline
=$ \frac{3}{4}$ . ((1 + 4) . $5^{k+1}$ - 1 )
\newline
=$  \frac{3 . (5^{k+1+1} - 1)}{4}$ 
\newline
so P(k+1) is true.
P(n) is true for all nonnegative integer n.
\problem{5: Mathematical Induction}{20}
Prove that $n^2$ - 1 is divisible by 8 whenever n is an odd
positive integer.
\solution
n = 2k + 1
\newline
$n^{2}-1  = (2k + 1)^{2} - 1 = 4k^{2} + 4k + 1 -1$
\newline
$4k^{2} + 4k $ = 4k(k + 1 )
\newline
k is odd, in which case (k+1) is even, and equal to 2q (q=non-negative integer)
\newline
if k is odd then , $n^{2}$ -1 = 4k(2q) = 8kq  so 8 divides $n^{2}-1$
\newline
if k is even k = 2q,  $n^{2}$ -1 = 4k(2q)(k + 1) = 8kq(k + 1) so 8 divides $n^{2}-1$
\newline
 See that 8 divides n²-1 in for all positive integer of n
\newline
\problem{6: Sets}{8}
Which of the following sets are equal? Show your work step by step.\newline
\subproblem{a} $\{$t : t is a root of $x^2$ – 6x + 8 = 0$\}$
\newline
\subproblem{b} $\{$y : y is a real number in the closed interval [2, 3]$\}$
\newline
\subproblem{c} $\{$4, 2, 5, 4$\}$
\newline
\subproblem{d} $\{$4, 5, 7, 2$\}$ - $\{$5, 7$\}$
\newline
\subproblem{e} $\{$q: q is either the number of sides of a rectangle or the number of digits in any integer between 11 and 99$\}$\\

\solution
\newline
Only a ,d ,e are equal.
\newline a=\{4,2\} beause t-4 = 0 , t-2 =0 
\newline
b = \{ 2 $<$= y $<$= 3\} 
\newline
d = A $\backslash$ B  =\{4,2\}
\newline e= \{4,2\} beacuse  x = 4 the number of sides of a rectangle ,x = 2 number of digits
\newpage
\problem{Bonus: Logic in Algorithms}{20}

Let p and q be the statements as follows.

\begin{itemize}
	\item $\textbf{p:}$ It is sunny.
	\item $\textbf{q:}$ The flowers are blooming.
\end{itemize}

\begin{figure*}[htp]
	\centering
	\includegraphics[scale=0.5]{circuit.png}
	\caption{Combinational Circuit}
	\label{fig: circuit}
	
\end{figure*}

In Figure \ref{fig: circuit}, the two statements are used as input. The circuit has 3 gates as AND, OR and NOT operators. It has also a 2x1 multiplexer\footnote{https://www.geeksforgeeks.org/multiplexers-in-digital-logic/} which provides to select one of the two options. 
\subproblem{a} Write the sentence that "result" output has.
\solution
\newline
(p $\lor$ $\neg$ q) 

\subproblem{b} Convert Figure \ref{fig: circuit} to an algorithm which you can write in any programming language that you prefer (including pseudocode).
\solution
\newline
int main()\{
\newline
int p ,q , result;
\newline
\newline
if((p == 1) \&\& (q == 0))
\newline
    result = 1;
\newline
else
\newline
    result = 0;
\newline
\}
\newline
\newline
\newline
\newline
\newline
\newline
\newline
\newline

\end{document} 